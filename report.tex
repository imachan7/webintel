\documentclass[a4paper,11pt]{ltjsarticle}
%
\usepackage{amsmath,amssymb}
\usepackage{bm}
\usepackage{physics}
\usepackage{graphicx}
\usepackage{ascmac}
\usepackage{cases}
\usepackage{here}
\usepackage{upgreek}
\usepackage{pgfgantt}
\usepackage[margin=1in]{geometry}
\usepackage{tcolorbox}
\usepackage{amsmath}
\usepackage{diagbox} % セル内に斜線を引くため
\usepackage{colortbl} % セルに色を付ける場合 (オプション)
\usepackage{multirow} % セルを縦に結合するため
\usepackage{url}
\usepackage{hyperref}
%
%
\title{高校倫理学習チャットボット説明書}
\author{35714020 今関蒼生}
\date{\today}
\begin{document}
\maketitle
%
%
\section{必須課題}
\subsection{プログラムの説明}
本プログラムは、高校倫理を学習分野とした学習支援AIチャットボットである。

\subsubsection{システム構成}
システム構成を以下に示す。

\begin{itemize}
\item \textbf{AIモデル}: Google Gemini 1.5 Flash を使用
\item \textbf{知識ベース}: terms.json ファイルに15カテゴリ(心理学、古代ギリシア哲学、キリスト教、仏教、儒教等)の用語情報を格納
\item \textbf{ペルソナ設定}: 高校倫理を極めた賢者の老人(一人称「わし」、古風な口調)
\item \textbf{機能}: カテゴリ選択、問題出題、正誤判定、段階的ヒント提供
\end{itemize}
Google Gemini 1.5 Flash を使用を使用しているのはAPIのクォータ制限になってしまい生成ができなくなってしまったので、
Google Gemini 1.5 Flash を使用している。

\subsubsection{主要機能}
主な機能を以下に示す。
\begin{enumerate}
\item \textbf{カテゴリ選択機能}: ユーザーが学習したいカテゴリを選択
\item \textbf{問題出題機能}: 選択されたカテゴリからランダムに用語を選択し、問題を生成
\item \textbf{正誤判定機能}: ユーザーの回答を正規化して正解と比較
\item \textbf{ヒント提供機能}: 不正解時に段階的なヒントを提供(答えは直接言わない)
\item \textbf{継続学習機能}: 正解後に同じカテゴリから次の問題を自動出題
\end{enumerate}
処理のフローとしては、ユーザーがカテゴリから勉強したい範囲を選択して、
そのカテゴリ内の用語をランダムに選択して、それが回答となるように問題を生成するようになっている。
また、ユーザーが回答を入力した際に、正規化を行い、正解と比較して正誤判定を行う。


\subsection{工夫点}
\subsubsection{ハルシネーション抑制の工夫}
最初にハルシネーションを抑制するための工夫について以下の3点を挙げる。
\begin{itemize}
\item \textbf{構造化された知識ベース}: JSONファイルに事前に検証済みの用語情報を格納し、これを基に問題を生成することで、不正確な情報の生成を抑制
\item \textbf{正解の事前設定}: プログラム側で正解を管理し、AIには問題生成のみを依頼することで、正誤判定の精度を確保
\item \textbf{プロンプト最適化}: システムインストラクションを簡潔にし、具体的な指示を与えることで、意図しない応答を減少
\end{itemize}
このjsonファイルの作成に関しては、
以下のリンク先の資料を基に作成した。
\url{
  http://haranonaka.com/download/ethics.html
}
こちらのリンク先のethical\_terms.pdfを参考にしている。
このjsonには人物、それに関する単語、その説明がふくまれている。
そして、これをプロンプトとして活用している。
これにより何もプロンプトを入れない状態で実行するよりもハルシネーションを抑制することができた。
また、プログラム側で正解と不正解を行っている。
これにより情報の正確性をある程度プログラム側で担保して、その説明という補完をAIに行わせている。

\subsubsection{技術的工夫}
\begin{itemize}
\item \textbf{テキスト正規化}: 全角・半角統一、記号除去等により、誤変換や表記揺れに対応
\item \textbf{エラーハンドリング}: API呼び出しやユーザー入力のエラーを適切に処理し、システムの安定性を向上
\item \textbf{処理の分離}: 正解応答と次問題出題を分離し、確実な問題提示を実現
\end{itemize}
エラーハンドリングに関してはよくわからない部分が多数あったため。AIに聞いたコードを参考にしている。
自分で実装したというよりは、AIに聞いたコードをそのまま実装したという感じである。
\subsection{実行例}
以下の実行例1に実際の実行例を示す:

\begin{tcolorbox}[title=実行例1]
\begin{verbatim}
model: わしじゃ!高校倫理を学ぶお手伝いをしよう 🦉
まずは、どのカテゴリから問題を出してほしいかね?
選択できるカテゴリ:
1. イスラム教
2. キリスト教
3. 中国思想
4. 仏教
...(省略)...
14. 近世哲学
カテゴリ名を直接入力してくれ。
user:キリスト教
model:イエス・キリストが説いた、神の愛に倣い、敵をも含む全ての人を平等に愛する精神を表す言葉は何かじゃ?🦉

user:隣人愛
model:見事じゃ!お見事じゃ!見事な解答じゃった!  🦉

「隣人愛」とは、まさにキリスト教の根幹をなす重要な概念じゃな。  単なる親切心や同情心とは異なり、神への愛から発する、全ての人 間に対する無償の愛を意味しておる。  イエス・キリストの教えの中心であり、新約聖書の中でも繰り返し強調されている重要なテーマじゃ。
豆知識じゃが、この「隣人愛」は、単に個人の善行にとどまらず、社会全体の調和や平和にも大きく貢献する概念として捉えられてきたん じゃ。  現代社会においても、様々な社会問題を考える上で、この「隣人愛」の精神は重要な指針となるであろう。  更に深く探求するな らば、隣人とは誰を指すのか?具体的な行動としてどのように表れるのか?などを考えてみるのも良いぞ。  そうすれば、倫理的な思考力 がより一層研ぎ澄まされるであろう。  🦉
\end{verbatim}
\end{tcolorbox}
このように、ユーザーがカテゴリを選択し、問題に回答することで学習を進めることができる。
また、正解した場合はそのまま次の問題が出題される。
以下の実行例2にその部分とそれが不正解でなかった場合も示す。
\begin{tcolorbox}[title=実行例2]
\begin{verbatim}
model:イエスの最初の弟子であり、エルサレム教会の礎を築き、熱心に福音を伝えた人物は誰じゃ?🦉

user:マリア
model:不正解じゃ! 🦉  マリアはイエスの母であり重要な人物じゃが、イエスの最初の弟子ではないぞ。

ヒントじゃ:この人物は、イエスに最初に呼ばれた弟子の一人であり、後にローマで殉教したと伝えられている。  名前は、新約聖書の中 でも頻繁に登場するぞ。  もう一度考えてみるが良い。 🦉

user:ペテロ
model:見事じゃ!お見事じゃ!見事な解答じゃった!  🦉

ペテロは、イエス・キリストの最初の弟子の一人であり、重要な役割を果たした人物じゃな。  彼の名前はシモン・ペトロ、もしくはケフ ァ(アラム語で「岩」の意味)とも呼ばれておる。  イエスは彼に「岩」というニックネームを与え、教会の礎を築く重要な人物として信 頼を置いたとされている。

豆知識じゃが...(省略)...
\end{verbatim}
\end{tcolorbox}
間違った答えを入力した場合は、ヒントを与えて再度回答を促す。
また、正解した場合はそのまま次の問題が出題されるようになっている。

\subsection{考察}
今回このチャットボットを作成して分かったことだが、初期のプロンプトで与える情報量や具体性が、生成される応答の質に大きく影響することが分かった。
特に、ユーザーの期待に応えるためには、明確で具体的な指示が重要だろう。
そして、突然多言語になることがあり、言語の一貫性を保つことが難しかった。
これはおそらくだが参照している情報源が多言語であるため、AIが混乱してしまうことが原因だと考えられる。


\subsection{自己評価}
\subsubsection{要件達成度}
\begin{itemize}
\item ✓ ヒントのみ提供(直接答えを言わない): 完全達成
\item ✓ AIペルソナ設定: 賢者の老人キャラクターで達成
\item ✓ 学習支援分野選択: 高校倫理で達成
\item ✓ 出題・正誤判定機能: 達成
\item ✓ 知識ベース構築: JSON形式で15カテゴリの用語データベース構築
\item ✓ ハルシネーション抑制: 構造化データと事前正解管理である程度達成
\item ✓発展課題A:jsonの説明にはない説明、ヒントが作成されるようになり達成
\end{itemize}

\subsubsection{総合評価}
要求仕様をすべて満たし、加えて会話記憶機能、エラーハンドリング、ユーザビリティ向上等の付加価値も実装できた。
特に「答えを教えずにヒントで導く」というコンセプトを技術的に実現し、実用的な学習支援ツールとして完成させることができた。


\section{最後に}
\subsection{感想}
今回の課題を通じて、AIチャットボットの設計と実装は大変だった。
プログラム的な面でもプロンプトを考えるという点でも難しさを感じた。
ある程度思い描いていた通りのボットにはなったのでよかった。
jsonファイルのもととなったデータに関しては感謝しかない。
最後になぜ急に多言語になるの?!
質問という単語がずっとロシア語になるときがあって
不思議だった。
\subsection{参考文献}
\begin{enumerate}
\item 概略理解 倫理用語集: \url{http://haranonaka.com/download/ethics.html}
\end{enumerate}



\end{document}